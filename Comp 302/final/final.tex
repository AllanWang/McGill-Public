\documentclass[12pt]{article}
    \usepackage{multicol}
    \usepackage{ifthen}
    \usepackage{geometry,amsmath}
    \usepackage{listings}
    \usepackage{color}
    \usepackage{enumitem}

    \pdfinfo{
      /Title (comp302_final.pdf)
      /Author (Allan Wang)
      /Subject (Comp 302)
      /Keywords (pdflatex, latex,pdftex,tex)}
    
    % This sets page margins to .5 inch if using letter paper, and to 1cm
    % if using A4 paper. (This probably isn't strictly necessary.)
    % If using another size paper, use default 1cm margins.
    \ifthenelse{\lengthtest { \paperwidth = 11in}}
        { \geometry{top=.5in,left=.5in,right=.5in,bottom=.5in} }
        {\ifthenelse{ \lengthtest{ \paperwidth = 297mm}}
            {\geometry{top=1cm,left=1cm,right=1cm,bottom=1cm} }
            {\geometry{top=1cm,left=1cm,right=1cm,bottom=1cm} }
        }
    
    % Turn off header and footer
    \pagestyle{empty}
   
	\definecolor{header}{rgb}{0.18,0.45,0.71}
    % Redefine section commands to 	use less space
    \makeatletter
    \renewcommand{\section}{\@startsection{section}{1}{0mm}%
                                    {-1ex plus -.5ex minus -.2ex}%
                                    {0.5ex plus .2ex}%x
                                    {\normalfont\large\bfseries\color{header}}}
    \renewcommand{\subsection}{\@startsection{subsection}{2}{0mm}%
                                    {-1explus -.5ex minus -.2ex}%
                                    {0.5ex plus .2ex}%
                                    {\normalfont\normalsize\bfseries}}
    \renewcommand{\subsubsection}{\@startsection{subsubsection}{3}{0mm}%
                                    {-1ex plus -.5ex minus -.2ex}%
                                    {1ex plus .2ex}%
                                    {\normalfont\small\bfseries}}
    \makeatother
    
    % Define BibTeX command
    \def\BibTeX{{\rm B\kern-.05em{\sc i\kern-.025em b}\kern-.08em
        T\kern-.1667em\lower.7ex\hbox{E}\kern-.125emX}}
    
    % Don't print section numbers
    \setcounter{secnumdepth}{0}
    
    
    \setlength{\parindent}{0pt}
    \setlength{\parskip}{0pt plus 0.5ex}
    
    %My Environments
    \newtheorem{example}[section]{Example}
    % -----------------------------------------------------------------------
    \lstset{language=[Objective]caml,flexiblecolumns=true}
    \newcommand{\oc}[1]{\lstinline{#1} \\}
    

\begin{document}
\raggedright
\footnotesize
\begin{multicols}{2}


	% multicol parameters
	% These lengths are set only within the two main columns
	%\setlength{\columnseprule}{0.25pt}
	\setlength{\premulticols}{1pt}
	\setlength{\postmulticols}{1pt}
	\setlength{\multicolsep}{10pt}
	\setlength{\columnsep}{2pt}



	\begin{center}
		\Large{\underline{Comp 302 Final}} \\
	\end{center}

	\section{Concepts}
	\begin{lstlisting}
let curry f = (fun x y -> f (x, y))
let curry2 f x y = f (x, y)
let curry3 = fun f -> fun x -> fun y -> f (x, y)
let uncurry f = (fun (x, y) -> f x y)

(* Functions are right associative *)
(* Functions are not evaluated until they need to be *)
let test a b = a * a + b
test 3 = fun y -> 3 * 3 + y (* Not 9 + y *)
	\end{lstlisting}


	% --------------------------------------------------------------------------
    \section{Syntax}
    \textcolor{red}{\textbf{Do not forget about 'rec', 'let ... in', brackets, constructors or tuples}}
	\begin{lstlisting}
match x with
  | a -> (* return *)
  | b -> (* Nested matching *)
    begin match ... with 
      | ... ->
    end
  | _ -> (* wildcard return *)
  
let name arg1 arg2 =
  let inner' arg1' arg2' = out' in
  inner' arg1 arg2
  
exception Failure of string
raise (Failure "what a terrible failure")

let x = 2 and y = 4 (* Initializes both *)

(* ('a * 'b -> 'c) -> 'a -> 'b -> 'c = <fun> *)
let cur = fun f -> fun x -> fun y -> f (x,y)

(* 'a list list -> 'a list = <fun> *)
let first lst = match lst with
  | [] -> [] | x::xs -> x

(* An anonymous 'function' has only one argument,
and can be matched directly without match ... with
val is_zero : int -> string = <fun> *)
let is_zero = function | 0 -> "zero" | _ -> "not zero"

(* Variable bindings are overshadowed; 
bindings are valid in their respective scopes *)
let m = 2;; let m = m * m in m (* is 4 *);; 
m (* is 2 *);; let f () = m;; let m = 3;; f () (* is 2 *);;
    \end{lstlisting}
    
	% --------------------------------------------------------------------------
	\section{List Ops}
	\begin{lstlisting}
elem :: list        list1 @ list2
val length : 'a list -> int
val find_opt : ('a -> bool) -> 'a list -> 'a option
val filter : ('a -> bool) -> 'a list -> 'a list 
val map : ('a -> 'b) -> 'a list -> 'b list
val fold_right : ('a -> 'b -> 'b) -> 'a list -> 'b -> 'b
val for_all : ('a -> bool) -> 'a list -> bool
val exists : ('a -> bool) -> 'a list -> bool
val rev : 'a list -> 'a list
val init : int -> (int -> 'a) -> 'a list (* by index *)
    \end{lstlisting}
    
    \columnbreak
    % --------------------------------------------------------------------------
    \section{Types \& Option}
    \begin{lstlisting}
(* Base types: bool, int, char, float, 'a list, option *)
(* 'x denotes a polymorphic type (Java Generics) *)
type 'a option = None | Some of 'a
(* Constructors can be used to match within types
match cases are sufficient once all constructors are matched *)
type rational = Integer of int
  | Fraction of rational * rational
type 'param int_pair = int * 'param
let x = (3, 3.14) (* val x : int * float = 3, 3.14 *)
(* Valid specified type *)
let (x : int_pair) = (3, 3.14) (* val x : int_pair = 3, 3.14 *)
    \end{lstlisting}
	
	% --------------------------------------------------------------------------
	\section{Higher Order Functions}
	\begin{lstlisting}
(* sum : (int -> int) -> int * int -> int *)
let rec sum f (a, b) =
  if a > b then 0
  else f a + sum f (a + 1, b)
(* sumCubes : int * int -> int = <fun> *)
let sumCubes (a, b) = sum (fun x -> x * x * x) (a, b)
	\end{lstlisting}

	% --------------------------------------------------------------------------
    \section{Induction}
    
    \begin{itemize}[noitemsep,nosep]
        \item Mathematical, Structural
        \item Can generalize theory when proving \\
        Eg rev (x::t) = rev' (x::t) [] $\Rightarrow$ rev l @ acc = rev' l acc
    \end{itemize} 
    \vspace{1mm}
	\begin{tabular}{l l}
	$e \Downarrow v$ 		& multi step evaluation from $e$ to $v$ 			\\
	$e \Rightarrow e'$ 		& single step evaluation from $e$ to $e'$ 			\\
	$e \Rightarrow^* e'$	& multiple small step evaluations from $e$ to $e'$ 	\\
    \end{tabular} \\~\\
    State theory and IH; do base case \\
    \begin{lstlisting}
let rec even_parity = function
  | [] -> false
  | true::xs -> not (even_parity xs)
  | false::xs -> even_parity xs

let even_parity_tr l = let rec parity p = function
  | [] -> p | p'::xs -> parity (p<>p') xs in
  parity false l

(* IH: For all l, even_parity l = even_parity_tr l *)   
(* Case for true: *)
even_parity_tr true::xs
= parity false true::xs      (* Def of even_parity_tr *)
= parity (false <> true) xs  (* Def of parity *)
= parity true xs             (* Def of <> *)
= not (parity false xs)      (* Prove? *)
= not (even_parity_tr xs)    (* Def of even_parity_tr *)
= not (even_parity xs)       (* IH *)
= even_parity true::xs       (* Def of even_parity *)

    \end{lstlisting}

    \section{References}
    \begin{tabular}{l l l}
        mutable type    & type t = \{ mutable i : int \} \\
        init        & let x = ref 0 & let y = { i = 5 } \\
        retrieval   & let val1 = !x & let val2 = y.i \\
        assignment  & x := 9    & y.i \textless \textendash 1

    \end{tabular}
	
\end{multicols}

\pagebreak

\begin{multicols}{2}

\begin{lstlisting}
module type STACK =
sig
    type stack
    type t
    val empty : unit -> stack
    val push : t -> stack -> stack
    val size: stack -> int
    val pop : stack -> stack option
    val peek : stack -> t option
end

module IntStack : (STACK with type t = int) =
struct
    type stack = int list
    type t = int
    let empty () = []
    let push i s = i :: s
    let size = List.length
    let pop = function | [] -> None | _ :: t -> Some t
    let peek = function | [] -> None | h :: _ -> Some h
end

(* Susp *)

type 'a susp = Susp of (unit -> 'a)
type 'a str = {hd: 'a; tl : ('a str) susp}
let delay f = Susp f      (* (unit -> 'a) -> 'a susp *)
let force (Susp f) = f()  (* 'a susp -> 'a *)

(* ('a -> 'b -> 'c) -> 'a str -> 'b str -> 'c str *)
let rec zip f s1 s2 = {hd = f s1.hd s2.hd ; 
    tl = delay (fun () -> zip f (force s1.tl) (force s2.tl)) }

(* Sieve of Eratosthenes *)

let rec filter_str (p : 'a -> bool) (s : 'a str) =
    let h, t = find_hd p s in {hd = h; 
    tl = delay (fun () -> filter_str p (force t))}
and find_hd p s = if p s.hd then (s.hd, s.tl)
    else find_hd p (force s.tl)

let no_divider m n = not (n mod m = 0)    
let rec sieve s = { hd = s.hd; 
    tl = delay (fun () -> 
        sieve (filter_str (no_divider s.hd) (force s.tl) ))}

(* val double : ('a -> 'a) -> 'a -> 'a = <fun> *)
let double = fun f -> fun x -> f(f(x))


\end{lstlisting}

\section{Misc}

An effect is an action resulting from evaluation of an expression other than returning a value.

\columnbreak

\begin{lstlisting}
(* Coin *)

exception BackTrack

(* val change : int list -> int -> int list = <fun> *)
let rec change coins amt = if amt = 0 then []
    else (match coins with 
        | [] -> raise BackTrack
        | coin :: cs ->
            if coin > amt then change cs amt
            else try coin :: (change coins (amt - coin)) 
                with BackTrack -> change cs amt)

(* val change : int list -> int -> 
    (int list -> 'a) -> (unit -> 'a) -> 'a = <fun> *)
let rec change coins amt success failure = 
    if amt = 0 then success []
    else match coins with
        | [] -> failure ()
        | coin :: cs -> 
            if coin > amt then change cs amt success failure
            else change coins (amt - coin)
                (fun list -> success (coin :: list))
                (fun () -> change cs amt success failure)
\end{lstlisting}

\section{Syntax \& Semantics}

\begin{itemize}[noitemsep, nosep]
    \item Definition \\
    $FV((e_1, e_2)) = FV(e_1) \cup FV(e_2)$
    \item Substitution \\
    $[e'/x'](\text{let pair } (x, y) = e_1 \text{ in } e_2 \text{ end}) = \text{let pair } (x, y) = [e'/x']e_1 \text{ in } [e'/x']e_2 \text{ end}$ \\
    Provided $x' \ne x \text{ \&\& } x' \ne y \text{ \&\& } (x, y \ne FV(e'))$
    \item Type \\
    Types $t ::= \text{int } \vert \text{ bool } \vert \, T_1 \rightarrow T_2 \vert \, T_1 \times T_2 \vert \, \alpha$ \\
    $\dfrac{\Gamma \vdash (x, y): T \times S \quad \Gamma \cup \{(x: T), (y: S)\} \vdash e_2: U}{\Gamma \vdash \text{let pair } (x, y) = e_1 \text{ in } e_2: U}$ T-LET \\
    Provided $x' \ne x, x' \ne y, \& \, x, y \notin FV(e')$ \\~\\
    My Solution (they don't account for $e_1$) \\

    $\dfrac{\Gamma \vdash e_1: T_1 \times T_2 \quad \Gamma, x: T_1, y: T_2, \vdash e_2: T}{\Gamma \vdash \text{ let } (x, y) = e_1 \text{ in } e_2 \text{ end}: T}$ \, T-LET-MATCH
    \item Operation \\
    $\dfrac{e_1 \Downarrow (v_1, v_2) \quad [v_1/x][v_2/y]e_3 \Downarrow v}{\text{let pair } (x, y) = e_1 \text{ in } e_2 \Downarrow v_3}$ B-LET
    \item References \\
    \vspace{1mm}
    $\dfrac{\Gamma \vdash e_1 : T \text{ ref} \quad \Gamma \vdash e_2 : T}{\Gamma \vdash e_1 := e_2 : \text{ unit}} \quad 
    \dfrac{\Gamma \vdash e : T \text{ ref}}{\Gamma \vdash !e : T}$ \\~\\
    $\dfrac{\Gamma \vdash e : T}{\Gamma \vdash \text{ref } e : T \text{ ref}} \quad
    \dfrac{}{\Gamma \vdash ()  : \text{ unit}}$
    \item Preservation: If $e \Downarrow v$ and $e : T$ then $v : T$
    
\end{itemize}
~\\
$\Gamma \vdash e \Rightarrow T/C$ : Infer type $T$ for expression $e$ in the typing environment $\Gamma$ module the constraints $C$
\vspace{1mm}
$\dfrac{x : T \in \Gamma}{\Gamma \vdash x \Rightarrow T/\emptyset}$ B-VAR
$\dfrac{\Gamma \vdash \Rightarrow T/C \quad \Gamma \vdash e_1 \Rightarrow T_1/C_1 \quad \Gamma \vdash e_2 \Rightarrow T_2/C_2}{\Gamma \vdash \text{if } e \text{ then } e_1 \text{ else } e_2 \Rightarrow T_1/C \cup C_1 \cup C_2 \cup \{T = \text{bool, } T_1 = T_2\}}$ B-IF

\end{multicols}

\end{document}