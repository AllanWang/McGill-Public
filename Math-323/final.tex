\documentclass[12pt]{article}
	\usepackage{hyperref, fancyhdr, setspace, enumerate}
	\usepackage{tabulary}
	\usepackage{amsmath, amsthm, amssymb, array, keycommand, lastpage, amssymb, xcolor, mathtools}
	\usepackage{multiaudience}
	\usepackage{tabularx}
	\usepackage{makecell}
	\usepackage{enumitem}
	\usepackage[margin=1 in]{geometry}
	\allowdisplaybreaks
	\hypersetup{
		%colorlinks=true, %set true if you want colored links
		linktoc=all, %set to all if you want both sections and subsections linked
		linkcolor=black, %choose some color if you want links to stand out
	}
	\author{Allan Wang} 
	\date{Last updated: \today}
	\title{MATH 323: Probability}
	\pagestyle{fancy}
	\lhead{MATH 323}
	\chead{\leftmark}
	\rhead{Allan Wang}
	\cfoot{Page \thepage \ of \pageref{LastPage}}
	
	% Only number for sections
	\setcounter{secnumdepth}{1}
	
	\newcommand\mm[1]{\begin{pmatrix}#1\end{pmatrix}}

	\setlength{\parindent}{0pt}
	
	\SetNewAudience{notes}
	\SetNewAudience{full}
	
	\setlist[enumerate]{itemsep=0mm}
	\setlist[itemize]{itemsep=0mm}

	\newcommand{\tab}[1]{\hspace{.2\textwidth}\rlap{#1}}
	
	\newcommand{\comment}[1]{}

	\newcommand{\mathcomment}[0]{\quad\color{blue}}

	\newcommand{\bigsum}[2]{\sum\limits_{#1}^{#2}}

	\newcommand{\ddef}[1]{\textcolor{blue}{#1}}
	
	\newcommand{\real}[0]{\mathbb{R}}
	
	\newcommand{\uu}[1]{\underbracket{#1}}

	\newkeycommand{\ccup}[sub=i=1, sup=\infty, base=A_i] {
		\bigcup_{\commandkey{sub}}^{\commandkey{sup}}\commandkey{base}
	}

	\newkeycommand{\ccap}[sub=i=1, sup=\infty, base=A_i] {
		\bigcap_{\commandkey{sub}}^{\commandkey{sup}}\commandkey{base}
	}

	\newkeycommand{\llim}[sub=n \rightarrow \infty, base=A_n] {
		\lim_{\commandkey{sub}}\commandkey{base}
	}

	\newkeycommand{\ssum}[sub=i=1, sup=k] {
		\sum_{\commandkey{sub}}^{\commandkey{sup}}	
	}

	\newenvironment{block}[1][Label]{\underline{#1}\par}{}
%	\newenvironment{proof}{\block[Proof]}{\endblock}
	\newenvironment{proposition}{\block[Proposition]}{\endblock}
	\newenvironment{lemma}{\block[Lemma]}{\endblock}
%	\newenvironment{theorem}{\block[Theorem]}{\endblock}
	\newenvironment{remark}{\block[Remark]}{\endblock}
	\newenvironment{definition}{\block[Definition]}{\endblock}

	\newcommand{\bb}[1]{\left\{#1\right\}}
	\newcommand{\pp}[1]{\left(#1\right)}
	\newcommand{\abs}[1]{\left|#1\right|}

	\newcommand{\divider}[0]{\par\textcolor{lightgray}{\rule{\textwidth}{0.1pt}}}
	
	\newenvironment{claim}{\textit{Claim:}}{\hfill $\square$}
	
	\newenvironment{remarks}{\underline{Remarks}\par}{}
	
	\newenvironment{example}{\shownto{-,notes}\underline{Example}\par}{\par\divider\endshownto}
	
	\newenvironment{eqn}{\equation\alignedat{3}}{\endalignedat\endequation}
	
	\newcommand{\todo}[0]{\textcolor{red}{\textbackslash\textbackslash TODO \ }}
	
	\newcounter{theorem}
	\newcommand{\theorem}[1]{\refstepcounter{theorem}\par\medskip
		\underline{Theorem~\thetheorem. #1}}
	
	\let\oldperp\perp
	\renewcommand{\perp}[0]{\oldperp\!\!\!\oldperp}

\begin{document}
\onehalfspacing
\maketitle
\tableofcontents
\pagebreak

\section{Formulas}

\subsection{Mean \& Variance}

\begin{tabularx}{\textwidth}{l | X | X}
	& Mean ($\mu$) & Variance ($\sigma^2$) \\
	General & $\overline{y} = \frac{1}{n} \sum_{i = 1}^n y_i$ & $\frac{1}{n - 1} \sum_{i = 1}^n (y_i - \overline{y})^2$
\end{tabularx}

\subsection{Combination \& Permutation}

\begin{itemize}
	\item $P_r^n = \frac{n!}{(n - r)!}$
	\item $\mm{n \\ r} = C_r^n = \frac{n!}{r! (n - r)!}$
	\item Permutation of $n$ objects of $k$ kinds, where $n_i$ is the number of times type $k$ occurs, is $\frac{n!}{n_1! n_2! ... n_k!}$
	\item Circular permutations (round table) with $n$ items taken $r$ at a time $ = \frac{P_r^n}{r}$
\end{itemize}

\subsection{Misc}

\begin{itemize}
	\item $P(A \mid B) = \frac{A \cap B}{P(B)}$ if $P(B) > 0$
\end{itemize}

\section{Theories \& Definitions}

\subsection{Kolmogorov Axioms}

$\forall A \in \Sigma$:

\begin{enumerate}
	\item $P(A) \ge 0$ (non-negative)
	\item $P(\Sigma) = 1$ (normed)
	\item $P \pp{\bigcup_{j = 1}^\infty A_j} = \sum_{j = 1}^\infty P(A_j)$ when $A_i \cap A_j = \emptyset \quad \forall i \ne j$ (linearly-additive)
\end{enumerate}

\subsection{De-Morgan's Theorem}

\begin{enumerate}
	\item $(A \cap B)^C = A^C \cup B^C$ 
	\item $(A \cup B)^C = A^C \cap B^C$
\end{enumerate}

\subsection{Partition}

Let $\bb{A_i}_{i = 1}^{\infty (k)}$ be a sequence of sets $A_i \le \Omega, \forall i$. If:

\begin{enumerate}
	\item $A_i \cap A_j = \emptyset, \forall i \ne j$ 
	\item $\bigcup_{i = 1}^{\infty (k)} A_i = \Sigma$
\end{enumerate}

then we say $\bb{A_i}_{i = 1}^{\infty (k)}$ is a partition of sample space $\Omega$: $\Omega = \bigcup_{j = 1}^k A_j$

\subsection{Inequalities (Boole, Bonferroni)}

\begin{enumerate}
	\item Boole's Inequality \\
	Let $A_i, A_2, ... \in \mathcal{F}$ \\
	$\Rightarrow P\pp{\bigcup_{i = 1}^\infty A_i} \le \sum_{i = 1}^\infty P(A_i)$ 
	\item Bonferroni's Inequality \\
	$P\pp{\bigcap_{i = 1}^\infty A_i} \ge 1 - \sum_{i = 1}^\infty P(A_i^C)$
\end{enumerate}

\subsection{Baye's Rule}

Assume $\bb{A_1, A_2, ..., A_k}$ is a partition of $\Omega$ such that $P(A_i) > 0 \forall i$ \\
For $B \subseteq \Omega$:

\begin{enumerate}
	\item Law of Total Probability \\
	\begin{eqn}
		P(B) &= P\pp{\bigcup_{i = 1}^k (A_i \cap B)} \\
		&= \sum_{i = 1}^k P(A_i \cap B) \\
		&= \sum_{i = 1}^k P(B \mid A_i) P(A_i) 
	\end{eqn}
	\item 
	\begin{eqn}
		P(A_i \mid B) &= \frac{P(A_i \cap B)}{P(B)} \\
		&= \frac{P(B \mid A_i) P(A_i)}{\sum_{i = 1}^k P(B \mid A_i) P(A_i)}
	\end{eqn}
\end{enumerate}

%-------------------------------------------------------------------------------

\section{Misc}

\begin{itemize}
	\item If $\bb{A_i}_{i = 1}^\infty$ is non-decreasing, then $\lim_{n \rightarrow \infty} A_n = \bigcup_{i = 1}^\infty A_i$
	\item If $\bb{A_i}_{i = 1}^\infty$ is non-increasing, then $\lim_{n \rightarrow \infty} A_n = \bigcap_{i = 1}^\infty A_i$
\end{itemize}

\subsection{Independence}

$A$ and $B$ are independent ($A \perp B$) if any of the following holds:

\begin{itemize}
	\item $P(A \mid B) = P(A)$ 
	\item $P(B \mid A) = P(B)$
	\item $P(A \cap B) = P(A)P(B)$
\end{itemize}

Note that $\emptyset$ is independent to any event, and that \\
$A \perp B \Leftrightarrow A^C \perp B \Leftrightarrow A^C \perp B^C \Leftrightarrow A \perp B^C$

\newpage
\section{Questions}
\begin{enumerate}
	\item How many ways can 4 married couples seat themselves around a circular table if no couple can sit next to each other?
\end{enumerate}

\newpage 
\section{Answers}
\begin{enumerate}
	\item $\frac{31}{105}$ - Compute probabilities that $i$ couple(s) sit next to each other for $i$ in $1..4$ and find complement of sum.
\end{enumerate}

\end{document}   