\documentclass[12pt]{article}
	\usepackage{hyperref, fancyhdr, setspace, enumerate, enumitem, amsmath, amsthm, amssymb, array, keycommand, lastpage, amssymb, xcolor}
	\usepackage[margin=1 in]{geometry}
	\allowdisplaybreaks
	\hypersetup{
		%colorlinks=true, %set true if you want colored links
		linktoc=all, %set to all if you want both sections and subsections linked
		linkcolor=black, %choose some color if you want links to stand out
	}
	\author{Allan Wang} 
	\date{Last updated: \today}
	\title{MATH 323: Probability}
	\pagestyle{fancy}
	\lhead{MATH 323}
	\chead{\leftmark}
	\rhead{Allan Wang}
	\cfoot{Page \thepage \ of \pageref{LastPage}}
	
	% Only number for sections
	\setcounter{secnumdepth}{1}

	\setlength{\parindent}{0pt}

	\newcommand{\tab}[1]{\hspace{.2\textwidth}\rlap{#1}}
	
	\newcommand{\comment}[1]{}

	\newcommand{\mathcomment}[0]{\quad\color{blue}}

	\newcommand{\bigsum}[2]{\sum\limits_{#1}^{#2}}

	\newcommand{\ddef}[1]{\textcolor{blue}{#1}}

	\newkeycommand{\ccup}[sub=i=1, sup=\infty, base=A_i] {
		\bigcup_{\commandkey{sub}}^{\commandkey{sup}}\commandkey{base}
	}

	\newkeycommand{\ccap}[sub=i=1, sup=\infty, base=A_i] {
		\bigcap_{\commandkey{sub}}^{\commandkey{sup}}\commandkey{base}
	}

	\newkeycommand{\llim}[sub=n \rightarrow \infty, base=A_n] {
		\lim_{\commandkey{sub}}\commandkey{base}
	}

	\newkeycommand{\ssum}[sub=i=i, sup=k] {
		\sum_{\commandkey{sub}}^{\commandkey{sup}}	
	}

	\newcommand{\braces}[1]{\left\{#1\right\}}

	\newcommand{\divider}[0]{\rule{\textwidth}{0.1pt}}
	
	\newenvironment{claim}{\textit{Claim:}}{\hfill $\square$}
	
	\newcommand{\todo}[0]{\textcolor{red}{\textbackslash\textbackslash TODO}}

\begin{document}
\onehalfspacing
\maketitle
\tableofcontents
\pagebreak
\section{2018/01/10}

\ddef{Experiment - process by which an observation is made}

\begin{itemize}
    \item Ex dice tossing is an experiment since we observe the number appearing on the other side
\end{itemize}

\section{2018/01/12}

\subsection{Limit of Sequence of Sets}

\ddef{Non-decreasing sequence - sequence of \(A_i\) such that \(A_j \subseteq A_k\) if \(i < k\)}

\ddef{Non-increasing sequence - sequence of \(A_i\) such that \(A_j \supseteq A_k\) if \(i > k\)}

\divider

Example 1

\begin{equation}
\begin{split}
	\text{Let } A_k & = \braces{x \mid 1 < x \le 2 - \frac{1}{k}} \\
	A_1 & = \braces{x \mid 1 < x \le 2 - \frac{1}{1}} \\
	A_2 & = \braces{x \mid 1 < x \le 2 - \frac{1}{2}} \\
	& \vdots \\
	A_k & = \braces{x \mid 1 < x \le 2 - \frac{1}{k}} \\
\end{split}
\end{equation}

\(\because\) the sequence is non-decreasing.

\(\therefore \llim = \ccup = \braces{x \mid 1 < x < 2}\)

Note that \(x < 2\) is open

\divider

Example 2

\(A_k = \braces{x \mid 1 < x \le x + \dfrac{1}{k}}\)

Note that the sequence is non-increasing.

\(\llim = \ccap = \braces{x \mid 1 < x \le 1} = \emptyset\)

Note that \(x \le 1\) is still closed.

\divider

\subsection{Partition \& Inequalities}

\ddef{Partition - sequence of mutually exclusive sets which together form the whole. More formally:}

Let \(\braces{A_i}_{i = 1}^{\infty (k)}\) be a sequence of sets

\(A_i \le \Omega, \forall i\)

\begin{tabular}{@{} l l}
If		& (a) \(A_i \cap A_j = \emptyset, \forall i \ne j\) \\
		& (b) \(\ccup[sup=\infty (k)] = \Omega \) \\
Then 	& $\braces{A_i}_{i=1}^{\infty(k)}$ is a partition of sample space $\Omega$
\end{tabular}

Note that if $B$ is any subset of $\Omega$ \& $\braces{A_i, ..., A_k}$ is a partition of $\Omega \Rightarrow B = \ccup[sup=k, base=(A_i \cap B)]$.

\divider

\subsubsection{Theorem 1. Kolmogorov Probability Measure}

Let \(F\) be a set that collects possible sets from \(\Omega\)

P: \(F \rightarrow R^+\) is called a probability measure if

\begin{enumerate}[label=(\alph*)]
	\item \(P(A) \ge 0 \quad \forall A \in F \text{non-negative}\)
	\item \(P(\Omega) = 1\)<br>
	\item \(P(\ccup) = \Sigma_{i=1}^{\infty} P(A_i), A_i \cap A_j = \emptyset \quad \forall i \ne j\)
\end{enumerate}

Some extensions:

\begin{enumerate}
	\item \(P(\phi) = 0\)
	\item \(P(\ccup[sup=k]) = \Sigma_{i=1}^{k} P(A_i), A_i \cap A_j = \emptyset \quad \forall i \ne j\)
	\item \(P(A^C) = 1 - P(A)\)
	\item \(A \subseteq B \Rightarrow P(A) \le P(B)\)
	\item \(P(A \cup B) = P(A) + P(B) - P(A \cap B)\)
	\item \(P(A_1 \cup A_2 \cup A_3) = P(A_1) + P(A_2) + P(A_3) - P(A_1 \cap A_2) - P(A_2 \cap A_3) - P(A_1 \cap A_3) + P(A_1 \cap A_2 \cap A_3)\)
\end{enumerate}

\begin{claim}
\begin{equation}
	\begin{alignedat}{3}
		&& P(\phi) &= 0 &\\
		\because \ && \Omega & = \Omega \cup \phi &\\
		\therefore \ && P(\Omega) & = P(\Omega \cup \phi) &\\
		&& & = P(\Omega) + P(\phi) & \mathcomment (\therefore \Omega \cap \phi = \phi) \\
		&& & = 1+ P(\phi) &\\
		\therefore \ && P(\phi) &= 0 &
	\end{alignedat}
\end{equation}
\end{claim}

\begin{claim}
	\begin{equation}
		\begin{alignedat}{3}
		&& P(A^C) &= 1 - P(A) &\\
		\because \ && \Omega & = A \cup A^C &\\
		\therefore \ && P(\Omega) &= P(A \cup A^C) &\\
		&& &	= P(A) + P(A^C) & \mathcomment (A \cap A^C = \phi) \\
		\therefore \ && 1 &= P(A) + P(A^C) &\\
		\therefore \ && P(A^C) &= 1 - P(A) &
		\end{alignedat}
	\end{equation}
\end{claim}

\begin{claim}
	\begin{equation}
		\begin{alignedat}{2}
			A \subseteq B & \Rightarrow P(A) \le P(B) &\\
			P(B) & = P(A \cup (A^C \cap B)) &\\
			& = P(A) + P(A^C \cap B) & \mathcomment (A \cap (A^C \cap B) = \emptyset) \\
			& \ge P(A) &
		\end{alignedat}
	\end{equation}
\end{claim}

\begin{claim}
	\begin{equation}
		\begin{alignedat}{3}
		&& P(A \cup B) &= P(A) + P(B) - P(A \cap B) &\\
		\because \ && A \cup B &= A \cup (A^C \cap B) &\\
		\therefore \ && P(A \cup B) &= P(A \cup (A^C \cap B)) & \mathcomment (A \cap (A^C \cap B) = \emptyset) \\
		&&&= P(A) + P(A^C \cap B) &\\
		\textit{Note} && P(B) &= P((A \cap B) \cup (A^C \cap B)) &\\
		&&&= P(A \cap B) + P(A^C \cap B) &\\
		\Rightarrow \ && P(A^C \cap B) &= P(B) - P(A \cap B) &
		\end{alignedat}
	\end{equation}
\end{claim}

//todo include other stuff

\subsection{Conditional Probability and Baye's Rule}

\ddef{The conditional probability of an event A, given that an event B has occurred, is equal to $P(A \mid B) = \frac{P(A \cap B)}{P(B)}$ if $P(B) > 0$}

Remark 1:

Here A is the event whose uncertainty we want to update, and B is the evidence we observe (or want to treat as given)

We call \(P(A)\) the prior probability of \(A\), and \(P(A \mid B)\) the posterior probability of A.)

\ddef{Prior - before updating based on evidence}

\ddef{Posterior - after updating based on evidence}

Remark 2:

\(P(A \mid B)\) is a probability measure

Check 1:

\(P(A \mid B) = \frac{P(A \cap B)}{P(B)} > 0\)

Since \(P(B) > 0\) and \(P(A \cap B) > 0\)

Check 2:

\(P(\Omega \mid B) = \frac{P(\Omega \cap B)}{P(B)} = \frac{P(B)}{P(B)} = 1\)

Check 3:

see check 3

\subsection{Bonferroni Inequalities}
Let $A_1, A_2, ..., \in F$ \\
$\Rightarrow P(\ccup) \le \ssum[sup=n] P(A_i)$

\todo

\begin{proof} \textit{1.}
	\begin{equation}
		\begin{alignedat}{2}
			&& \ccup[sup=k] &= A_i \cup (A^C_1 \cap A_2) \cup (A^C_1 \cap A^C_2 \cap A_3) \cup ... \cup (A^C_1 \cap ... \cap A^C_{k-1} \cap A_k) \\
			\Rightarrow \ && P(\ccup[sup=k]) &= P(A_1) + P(A^C_1 \cap A_2) + P(A^C_1 \cap A^C_2 \cap A_3) + ... \\
			&&& \le P(A_1) + P(A_2) + P(A_3) + ... P(A_k) \\
			&&&= \ssum P(A_i)
		\end{alignedat}
	\end{equation}
\end{proof}

\begin{proof}
	\begin{equation}
		\begin{alignedat}{3}
		\textit{Let } && B^C_i &= A_i &\\
		&& P(\ccap) &= P(\ccap[base=B^C_i]) &\\
		&& &= P((\ccup[base=B_i])^C) & \mathcomment \text{by DeMorgan} \\
		&& &= 1 - P(\ccup[base=B_i]) &\\
		&& & \ge 1 - \ssum[sup=\infty] P(B_i) & \mathcomment \text{by Boole's Inequality} \\
		&& &= 1 - \ssum[sup=\infty]P(A^C_i)
		\end{alignedat}
	\end{equation}
\end{proof}

\end{document}