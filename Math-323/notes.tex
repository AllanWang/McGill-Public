\documentclass[12 pt]{article}
	\usepackage{hyperref, fancyhdr, setspace, enumerate, amsmath,
	  lastpage, amssymb, xcolor, parskip}
	\usepackage[margin=1 in]{geometry}
	\allowdisplaybreaks
	\hypersetup{
		%colorlinks=true, %set true if you want colored links
		linktoc=all,     %set to all if you want both sections and subsections linked
		linkcolor=black,  %choose some color if you want links to stand out
	}
	\usepackage{graphicx}
	\graphicspath{{Images/}}
	\author{Allan Wang}
	\date{Last updated: \today}
	\title{MATH 323: Probability}
	\pagestyle{fancy}
	\lhead{MATH 323}
	\chead{\leftmark}
	\rhead{Allan Wang}
	\cfoot{Page \thepage \ of \pageref{LastPage}}
	\newcommand{\tab}[1]{\hspace{.2\textwidth}\rlap{#1}}
	
	\newcommand{\cupinfty}[0]{\cup_{i=1}^{\infty}A_i}
	\newcommand{\cupinftyup}[1]{\cup_{i=1}^{#1}A_i}
	\newcommand{\capinfty}[0]{\cap_{i=1}^{\infty}A_i}
	\newcommand{\capinftyup}[1]{\cap_{i=1}^{#1}A_i}
	\newcommand{\limitinfty}[0]{\lim_{n \rightarrow \infty}A_n}
	\newcommand{\mathcomment}[1]{\qquad\color{blue}{\(#1\)}}
	\newcommand{\bigsum}[2]{\sum\limits_{#1}^{#2}}
	\newcommand{\comment}[1]{}
	\newcommand{\definition}[1]{\textcolor{blue}{#1}}
	\begin{document}
		\onehalfspacing
		\maketitle
		\tableofcontents
			\section{2018/01/10}
			
			Midterm is on Mar 14th
			
			\subsection{Chapter 1 - Probability Introduction}
			
			\definition{Experiment - process by which an observation is made}
			
			Ex dice tossing is an experiment since we observe the number appearing on the other side
			
			Random experiment - TODO
			
			\definition{Sample space - S or Ω - collection of every possible outcome}
			
			\definition{Event - A, B, C - partial collection of sample space}
			
			Let $P(A) = \text{ probability of } A \ \forall A, A \in \Omega P(A)$ has the following axioms:
			
			$P(A) \ge 0$
			$P(S) = 1$ (unit measure / normed)
			if $i \ne j$ and $A_i \cap A_j = \emptyset$, $P(A_i \cup A_j) = P(A_i) + P(A_j)$
			
			\section{2018/01/12}
			Limit of a sequence of sets
			
			Limit of Sequence of Sets
			Non-decreasing sequence - sequence of $A_i$ such that $A_j \subseteq A_k$ if $i < k$
			
			Non-increasing sequence - sequence of $A_i$ such that $A_j \supseteq A_k$ if $i > k$
			
			Example 1
			\comment{
			\begin{align} \text{Let } A_k & = \braces{x \mid 1 < x \le 2 - \dfrac{1}{k}} \ A_1 & = \braces{x \mid 1 < x \le 2 - \dfrac{1}{1}} \ A_2 & = \braces{x \mid 1 < x \le 2 - \dfrac{1}{2}} \ & ... \end{align}
			
			Note that the sequence is non-decreasing.
			
			$\limitinfty = \cupinfty = \braces{x \mid 1 < x < 2}$
			
			Note that $x < 2$ is open
			
			Example 2
			
			$A_k = \braces{x \mid 1 < x \le x + \dfrac{1}{k}}$
			
			Note that the sequence is non-increasing.
			
			$\limitinfty = \capinfty = \braces{x \mid 1 < x \le 1} = \emptyset$
			
			Note that $x \le 1$ is still closed.
			
			1.4 Partition & Inequalities
			Partition - sequence of mutually exclusive sets which together form the whole. More formally:
			
			Let $\braces{A_i}_{i = 1}^{\infty (k)}$ be a sequence of sets
			
			$A_i \le \Omega, \forall i$
			
			If $A_i \cap A_j = \emptyset, \forall i \ne j$ and $\cupinftyup{\infty (k)} = \Omega$
			then $\braces{A_i}_{i=1}^{\infty (k)}$ is a partition of sample space $\Omega$
			
			Note - if $B$ is any subset of $\Omega$ & $\braces{A_1, ..., A_k}$ is a partition of $\Omega \Rightarrow B = \cupinftyup{k} (A_i \cap B)$
			
			Theorem 1. Kolmogorov Probability Measure
			Let $F$ be a set that collects possible sets from $\Omega$
			
			P: $F \rightarrow R^+$ is called a probability measure if
			
			a. $P(A) \ge 0 \quad \forall A \in F \mathcomment{\text{non-negative}}$
			b. $P(\Omega) = 1$
			c. $P(\cupinfty) = \Sigma_{i=1}^{\infty} P(A_i), A_i \cap A_j = \emptyset \quad \forall i \ne j$
		}
			Some extensions:
			
			$P(\phi) = 0$
			$P(\cupinftyup{k}) = \Sigma_{i=1}^{k} P(A_i), A_i \cap A_j = \emptyset \quad \forall i \ne j$
			$P(A^C) = 1 - P(A)$
			$A \subseteq B \Rightarrow P(A) \le P(B)$
			$P(A \cup B) = P(A) + P(B) - P(A \cap B)$
			$P(A_1 \cup A_2 \cup A_3) = P(A_1) + P(A_2) + P(A_3) - P(A_1 \cap A_2) - P(A_2 \cap A_3) - P(A_1 \cap A_3) + P(A_1 \cap A_2 \cap A_3)$
			Proof
			
			Claim: $P(\phi) = 0$
			
			$\because \Omega = \Omega \cup \phi$
			
			$\therefore P(\Omega) = P(\Omega \cup \phi)$
			
			$= P(\Omega) + P(\phi) (\therefore \Omega \cap \phi = \phi)$
			
			$= 1+ P(\phi)$
			
			$\therefore P(\phi) = 0$
			
			Claim: $P(A^C) = 1 - P(A)$
			
			$\because \Omega = A \cup A^C$
			
			$\therefore P(\Omega) = P(A \cup A^C)$
			
			$= P(A) + P(A^C) --- (A \cap A^C = \phi)$
			
			$\therefore 1 = P(A) + P(A^C)$
			
			$\therefore P(A^C) = 1 - P(A)$
			
			Claim: $A \subseteq B \Rightarrow P(A) \le P(B)$
			
			$P(B) = P(A \cup (A^C \cap B))$
			
			$= P(A) + P(A^C \cap B) --- (A \cap (A^C \cap B) = \emptyset)$
			
			$\ge P(A)$
			
			Claim: $P(A \cup B) = P(A) + P(B) - P(A \cap B)$
			
			$\because A \cup B = A \cup (A^C \cap B)$
			
			$\therefore P(A \cup B) = P(A \cup (A^C \cap B)) --- (A \cap (A^C \cap B) = \emptyset)$
			
			$= P(A) + P(A^C \cap B)$
			
			Note: $P(B) = P((A \cap B) \cup (A^C \cap B))$
			
			$= P(A \cap B) + P(A^C \cap B)$
			
			$\Rightarrow P(A^C \cap B) = P(B) - P(A \cap B)$
			
			//todo include other stuff
			
			1.5 Conditional Probability and Baye's Rule
			
			Def 1 The conditional probability of an event A, given that an event B has occurred, is equal to
			
			$P(A \mid B) = \dfrac{P(A \cap B)}{P(B)} if P(B) > 0$
			
			Remark 1:
			
			Here A is the event whose uncertainty we want to update, and B is the evidence we observe (or want to treat as given
			
			We call P(A) the prior probability of A, and P(A | B) the posterior probability of A.)
			
			Prior - before updating based on evidence
			
			Posterior - after updating based on evidence
			
			Remark 2:
			
			$P(A \mid B)$ is a probability measure
			
			Check 1:
			
			$P(A \mid B) = \dfrac{P(A \cap B)}{P(B)} > 0$
			
			Since $P(B) > 0$ and $P(A \cap B) > 0$
			
			Check 2:
			
			$P(\Omega \mid B) = \dfrac{P(\Omega \cap B)}{P(B)} = \dfrac{P(B)}{P(B)} = 1$
			
			Check 3:
			
			see check 3
	\end{document}
	