\documentclass[12pt]{article}
	\usepackage{hyperref, fancyhdr, setspace, enumerate, amsmath, amsthm, amssymb, array, keycommand, lastpage, amssymb, xcolor}
	\usepackage{multiaudience}
	\usepackage{enumitem}
	\usepackage[margin=1 in]{geometry}
	\allowdisplaybreaks
	\hypersetup{
		%colorlinks=true, %set true if you want colored links
		linktoc=all, %set to all if you want both sections and subsections linked
		linkcolor=black, %choose some color if you want links to stand out
	}
	\author{Allan Wang} 
	\date{Last updated: \today}
	\title{MATH 323: Probability}
	\pagestyle{fancy}
	\lhead{MATH 323}
	\chead{\leftmark}
	\rhead{Allan Wang}
	\cfoot{Page \thepage \ of \pageref{LastPage}}
	
	% Only number for sections
	\setcounter{secnumdepth}{1}
	
	\newcommand\mm[1]{\begin{pmatrix}#1\end{pmatrix}}

	\setlength{\parindent}{0pt}
	
	\SetNewAudience{notes}
	\SetNewAudience{full}

	\newcommand{\tab}[1]{\hspace{.2\textwidth}\rlap{#1}}
	
	\newcommand{\comment}[1]{}

	\newcommand{\mathcomment}[0]{\quad\color{blue}}

	\newcommand{\bigsum}[2]{\sum\limits_{#1}^{#2}}

	\newcommand{\ddef}[1]{\textcolor{blue}{#1}}

	\newkeycommand{\ccup}[sub=i=1, sup=\infty, base=A_i] {
		\bigcup_{\commandkey{sub}}^{\commandkey{sup}}\commandkey{base}
	}

	\newkeycommand{\ccap}[sub=i=1, sup=\infty, base=A_i] {
		\bigcap_{\commandkey{sub}}^{\commandkey{sup}}\commandkey{base}
	}

	\newkeycommand{\llim}[sub=n \rightarrow \infty, base=A_n] {
		\lim_{\commandkey{sub}}\commandkey{base}
	}

	\newkeycommand{\ssum}[sub=i=1, sup=k] {
		\sum_{\commandkey{sub}}^{\commandkey{sup}}	
	}

	\newcommand{\bb}[1]{\left\{#1\right\}}
	\newcommand{\pp}[1]{\left(#1\right)}

	\newcommand{\divider}[0]{\rule{\textwidth}{0.1pt}}
	
	\newenvironment{claim}{\textit{Claim:}}{\hfill $\square$}
	
	\newenvironment{remarks}{\underline{Remarks}\par}{}
	
	\newenvironment{example}{\shownto{-,notes}\underline{Example}\par}{\newline\divider\endshownto}
	
	\newenvironment{eqn}{\equation\alignedat{3}}{\endalignedat\endequation}
	
	\newcommand{\todo}[0]{\textcolor{red}{\textbackslash\textbackslash TODO}}
	
	\newcounter{theorem}
	\newcommand{\theorem}[1]{\refstepcounter{theorem}\par\medskip
		\underline{Theorem~\thetheorem. #1}}
	
	\let\oldperp\perp
	\renewcommand{\perp}[0]{\oldperp\!\!\!\oldperp}

\begin{document}
\onehalfspacing
\maketitle
\tableofcontents
\pagebreak
\section{2018/01/10}

\ddef{Experiment - process by which an observation is made}

\ddef{Random Experiment - experiment where:}

\begin{itemize}
	\item A collection of every possible outcome can be observed prior to its performance
	\item It can be repeated under the same condition
\end{itemize}

\ddef{Samples Space $(S, \Omega)$ - collection of every possible outcome in an experiment}

\begin{itemize}
    \item Ex dice tossing is an experiment since we observe the number appearing on the other side
\end{itemize}

\ddef{Kolmogorov Axioms - for every event $A$ in $\Omega$, the probability of an event $P(A)$ satisfied}

\begin{enumerate}
	\item $P(A) \ge 0$
	\item $P(\Omega) = 1$
	\item $P(\ccup) = \ssum[sup=\infty] P(A_i)$ when $A_i \cap A_j = \emptyset \ \forall i \ne j$
\end{enumerate}

$$P(A) = \frac{n(A)}{n(\Omega)}$$

\section{2018/01/12}

\subsection{Limit of Sequence of Sets}

\ddef{Non-decreasing sequence - sequence of \(A_i\) such that \(A_j \subseteq A_k\) if \(i < k\)}

\ddef{Non-increasing sequence - sequence of \(A_i\) such that \(A_j \supseteq A_k\) if \(i > k\)}

\divider

\underline{Example 1}
\begin{eqn}
	\text{Let } A_k & = \bb{x \mid 1 < x \le 2 - \frac{1}{k}} \\
	A_1 & = \bb{x \mid 1 < x \le 2 - \frac{1}{1}} \\
	A_2 & = \bb{x \mid 1 < x \le 2 - \frac{1}{2}} \\
	& \vdots \\
	A_k & = \bb{x \mid 1 < x \le 2 - \frac{1}{k}} \\
\end{eqn}

\(\because\) the sequence is non-decreasing.

\(\therefore \llim = \ccup = \bb{x \mid 1 < x < 2}\)

Note that \(x < 2\) is open

\divider

\underline{Example 2} \\

\(A_k = \bb{x \mid 1 < x \le x + \dfrac{1}{k}}\)

Note that the sequence is non-increasing.

\(\llim = \ccap = \bb{x \mid 1 < x \le 1} = \emptyset\)

Note that \(x \le 1\) is still closed.

\divider

\subsection{Partition \& Inequalities}

\ddef{Partition - sequence of mutually exclusive sets which together form the whole. More formally:}

Let \(\bb{A_i}_{i = 1}^{\infty (k)}\) be a sequence of sets

\(A_i \le \Omega, \forall i\)

\begin{tabular}{@{} l l}
If		& (a) \(A_i \cap A_j = \emptyset, \forall i \ne j\) \\
		& (b) \(\ccup[sup=\infty (k)] = \Omega \) \\
Then 	& $\bb{A_i}_{i=1}^{\infty(k)}$ is a partition of sample space $\Omega$
\end{tabular}

Note that if $B$ is any subset of $\Omega$ \& $\bb{A_i, ..., A_k}$ is a partition of $\Omega \Rightarrow B = \ccup[sup=k, base=(A_i \cap B)]$.

\divider

\theorem{Kolmogorov Probability Measure}
	
Let \(F\) be a set that collects possible sets from \(\Omega\)

P: \(F \rightarrow R^+\) is called a probability measure if

\begin{enumerate}[label=(\alph*)]
	\item \(P(A) \ge 0 \quad \forall A \in F \text{non-negative}\)
	\item \(P(\Omega) = 1\)<br>
	\item \(P(\ccup) = \Sigma_{i=1}^{\infty} P(A_i), A_i \cap A_j = \emptyset \quad \forall i \ne j\)
\end{enumerate}

Some extensions:

\begin{enumerate}
	\item \(P(\phi) = 0\)
	\item \(P(\ccup[sup=k]) = \Sigma_{i=1}^{k} P(A_i), A_i \cap A_j = \emptyset \quad \forall i \ne j\)
	\item \(P(A^C) = 1 - P(A)\)
	\item \(A \subseteq B \Rightarrow P(A) \le P(B)\)
	\item \(P(A \cup B) = P(A) + P(B) - P(A \cap B)\)
	\item \(P(A_1 \cup A_2 \cup A_3) = P(A_1) + P(A_2) + P(A_3) - P(A_1 \cap A_2) - P(A_2 \cap A_3) - P(A_1 \cap A_3) + P(A_1 \cap A_2 \cap A_3)\)
\end{enumerate}

\begin{claim}
\begin{eqn}
		&& P(\phi) &= 0 &\\
		\because \ && \Omega & = \Omega \cup \phi &\\
		\therefore \ && P(\Omega) & = P(\Omega \cup \phi) &\\
		&& & = P(\Omega) + P(\phi) & \mathcomment (\therefore \Omega \cap \phi = \phi) \\
		&& & = 1+ P(\phi) &\\
		\therefore \ && P(\phi) &= 0 &
\end{eqn}
\end{claim}

\begin{claim}
	\begin{eqn}
		&& P(A^C) &= 1 - P(A) &\\
		\because \ && \Omega & = A \cup A^C &\\
		\therefore \ && P(\Omega) &= P(A \cup A^C) &\\
		&& &	= P(A) + P(A^C) & \mathcomment (A \cap A^C = \phi) \\
		\therefore \ && 1 &= P(A) + P(A^C) &\\
		\therefore \ && P(A^C) &= 1 - P(A) &
	\end{eqn}
\end{claim}

\begin{claim}
	\begin{eqn}
			A \subseteq B & \Rightarrow P(A) \le P(B) &\\
			P(B) & = P(A \cup (A^C \cap B)) &\\
			& = P(A) + P(A^C \cap B) & \mathcomment (A \cap (A^C \cap B) = \emptyset) \\
			& \ge P(A) &
	\end{eqn}
\end{claim}

\begin{claim}
	\begin{eqn}
		&& P(A \cup B) &= P(A) + P(B) - P(A \cap B) &\\
		\because \ && A \cup B &= A \cup (A^C \cap B) &\\
		\therefore \ && P(A \cup B) &= P(A \cup (A^C \cap B)) & \mathcomment (A \cap (A^C \cap B) = \emptyset) \\
		&&&= P(A) + P(A^C \cap B) &\\
		\textit{Note} && P(B) &= P((A \cap B) \cup (A^C \cap B)) &\\
		&&&= P(A \cap B) + P(A^C \cap B) &\\
		\Rightarrow \ && P(A^C \cap B) &= P(B) - P(A \cap B) &
	\end{eqn}
\end{claim}

\subsection{Conditional Probability and Baye's Rule}

\ddef{The conditional probability of an event A, given that an event B has occurred, is equal to $P(A \mid B) = \frac{P(A \cap B)}{P(B)}$ if $P(B) > 0$}

Remark 1:

Here A is the event whose uncertainty we want to update, and B is the evidence we observe (or want to treat as given)

We call \(P(A)\) the prior probability of \(A\), and \(P(A \mid B)\) the posterior probability of A.)

\ddef{Prior - before updating based on evidence}

\ddef{Posterior - after updating based on evidence}

Remark 2:

\(P(A \mid B)\) is a probability measure

Check 1:

\(P(A \mid B) = \frac{P(A \cap B)}{P(B)} > 0\)

Since \(P(B) > 0\) and \(P(A \cap B) > 0\)

Check 2:

\(P(\Omega \mid B) = \frac{P(\Omega \cap B)}{P(B)} = \frac{P(B)}{P(B)} = 1\)

Check 3:

see check 3

\subsection{Bonferroni Inequalities}
Let $A_1, A_2, ..., \in F$ \\
$\Rightarrow P(\ccup) \le \ssum[sup=n] P(A_i)$

\begin{proof} \textit{1.}
	\begin{equation}
		\begin{alignedat}{2}
			&& \ccup[sup=k] &= A_i \cup (A^C_1 \cap A_2) \cup (A^C_1 \cap A^C_2 \cap A_3) \cup ... \cup (A^C_1 \cap ... \cap A^C_{k-1} \cap A_k) \\
			\Rightarrow \ && P(\ccup[sup=k]) &= P(A_1) + P(A^C_1 \cap A_2) + P(A^C_1 \cap A^C_2 \cap A_3) + ... \\
			&&& \le P(A_1) + P(A_2) + P(A_3) + ... P(A_k) \\
			&&&= \ssum P(A_i)
		\end{alignedat}
	\end{equation}
\end{proof}

\begin{proof}
	\begin{equation}
		\begin{alignedat}{3}
		\textit{Let } && B^C_i &= A_i &\\
		&& P(\ccap) &= P(\ccap[base=B^C_i]) &\\
		&& &= P((\ccup[base=B_i])^C) & \mathcomment \text{by DeMorgan} \\
		&& &= 1 - P(\ccup[base=B_i]) &\\
		&& & \ge 1 - \ssum[sup=\infty] P(B_i) & \mathcomment \text{by Boole's Inequality} \\
		&& &= 1 - \ssum[sup=\infty]P(A^C_i)
		\end{alignedat}
	\end{equation}
\end{proof}

\section{2018/01/17}

\ddef{Independence between two events $A$ and $B$, denoted as $A \perp B$, is when:}

\begin{itemize}
	\item $P(A \mid B) = P(A)$
	\item $P(B \mid A) = P(B)$
	\item $P(A \cap B) = P(A) P(B)$
\end{itemize}

The following statements are equivalent

\begin{enumerate}
	\item $A \perp B \Leftrightarrow A^C \perp B$
	\item $A \perp B \Leftrightarrow A^C \perp B^C$
	\item $A \perp B \Leftrightarrow A \perp B^C$
\end{enumerate}

\theorem{Baye's Rule}

Assume $\bb{A_1, A_2, ..., A_k}$ is a partition of $\Omega$ such that $P(A_i) > 0 \ \forall i$. For $B \subseteq \Omega$, we have

\begin{enumerate}[label=(\alph*)]
	\item Law of total probability 
	\begin{eqn}
		P(B) &= P \pp{\ccup[sup=k, base=(A_i \cap B)]} \\
		&= \ssum P(A_i \cap B) \\
		&= \ssum P(B \mid A_i) P(A_i)
	\end{eqn}

	\item \begin{eqn}
		P(A_i \mid B) &= \frac{A_i \cap B}{P(B)} \\
		&= \frac{P(B \mid A_i) P(A_i)}{\ssum P(B \mid A_i) P(A_i)}
	\end{eqn}
\end{enumerate}

\begin{example} 
	(Baye's Rule) \\
	You have two coins, one fair and one biased with a $\frac{3}{4}$ probability of landing on heads. You pick a coin at random and flip it three times, resulting in three heads. What is the probability that the coin is fair? \\
	
	\underline{Solution}
	
	\begin{itemize}
		\item Let $A_1 = P(\text{fair coin})$
		\item Let $A_2 = P(\text{biased coin})$
		\item Let $B = P(\text{fair coin \& three heads})$
	\end{itemize}
	
	\begin{eqn}
		P(A_1 \mid B) &= \frac{P(A_1 \cap B)}{P(B)} \\
		&= \frac{P(A_1) P(B \mid A_1)}{P(B)} \\
		&= \frac{P(A_1) P(B \mid A_1)}{P(A_1) P(B \mid A_1) + P(A_2) P(B \mid A_2)} \\
		&= \frac{\frac{1}{2} \cdot \pp{\frac{1}{2}}^3}{\frac{1}{2} \cdot \pp{\frac{1}{2}}^3 + \frac{1}{2} \cdot \pp{\frac{3}{4}}^3} \\
		&\approx 0.23		
	\end{eqn}
\end{example}

\section{2018/01/19}

\begin{remarks}
	\begin{enumerate}
		\item Before flipping the coin, we thought we were equally likely to have picked the fair coin as the biased coin, ie $P(A) = P(A^C) = 0.5$ \\
		However, upon observing three heads, it becomes more likely that we have chosen the biased coin, as $P(A^C \mid B) = 1 - 0.23 = 0.77$
		
		\item It is incorrect to say that $P(A) = 1$, as $P(A)$ is the probability prior to the observation, and we have merely observed that $P(A \mid A) = 1$
	\end{enumerate}
\end{remarks}

\begin{example}
	\begin{itemize}
		\item Bowl $A_1$: 3 red \& 7 blue chips
		\item Bowl $A_2$: 8 red \& 2 blue chips
	\end{itemize}

	Tossing a die, bowl $A_1$ is selected if a 5 or a 6 is rolled. Otherwise, $A_2$ is selected. \\
	Find the probability that the chip is from $A_1$ given that it is red ($P(A_1 \mid \text{Red})$)
	
	\begin{equation}
		\begin{alignedat}{2}
			& P(A_1) &&= \frac{n(5 \text{ or } 6)}{n(\Omega)} = \frac{2}{6} \\
			& P(A_2) &&= \frac{4}{6} \\
			& P(\text{Red} \mid A_1) &&= \frac{3}{10} \\
			& P(\text{Red} \mid A_2) &&= \frac{8}{10} \\
			& P(A_1 \mid \text{Red}) &&= \frac{P(A_1 \cap A)}{P(A)} \\
			& &&= \frac{P(A_1) \cdot P(A \mid A_1)}{P(A_1) \cdot P(A \mid A_1) + P(A_2) \cdot P(A \mid A_2)} \\
			& &&= \frac{\frac{2}{6} \cdot \frac{3}{10}}{\frac{2}{6} \cdot \frac{3}{10} + \frac{4}{6} \cdot \frac{8}{10}} \\
			& &&= \frac{3}{19}
		\end{alignedat}
	\end{equation}
\end{example}

\subsection{Permutation \& Combination}

\begin{example}
	What's the probabiltiy that 20 people have different birthdays? Assume that the year has 365 days
	
	\begin{eqn}
		& P(A) &&= \frac{n(A)}{n(\Omega)} \\
		& n(\Omega) &&= 365 \times 365 \times 365 ... \times 365 \\
		& n(A) &&= 365 \times 364 \times 363 \times ... 346 \\
		\therefore \ & P(A) &&= \frac{365 \times 364 \times ... \times 346}{365^{20}}
	\end{eqn}
\end{example}

\ddef{Permutation - An ordered arrangement of $r$ distinct objects from a set of $n$, denoted as $P^n_r$}

\begin{eqn}
	P^n_r = \frac{n!}{(n - r)!}
\end{eqn}

\begin{example} 
	There are 6 people on the bus, and 10 stops remaining. What's the probability that each person gets off at a different stop?
\end{example}

\ddef{Combination - An unordered arrangement of $r$ distinct objects from a set of $n$, denoted as $\mm{n \\ r} = C^n_r =\frac{n!}{r! (n - r)!}$} \\

\begin{example}
	How many different choices of drawing 5 cards from a standard deck of 52 playing cards?
\end{example}

\begin{example}
	How many different permutations are there of the letters in the word "book"?
\end{example}

\end{document}