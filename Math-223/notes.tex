\documentclass[12pt]{article}
	\usepackage{hyperref, fancyhdr, setspace, enumerate, amsmath, array, keycommand, lastpage, amssymb, xcolor}
	\usepackage[margin=1 in]{geometry}
	\allowdisplaybreaks
	\hypersetup{
		%colorlinks=true, %set true if you want colored links
		linktoc=all, %set to all if you want both sections and subsections linked
		linkcolor=black, %choose some color if you want links to stand out
	}
	\author{Allan Wang} 
	\date{Last updated: \today}
	\title{MATH 223: Linear Algebra 2}
	\pagestyle{fancy}
	\lhead{MATH 223}
	\chead{\leftmark}
	\rhead{Allan Wang}
	\cfoot{Page \thepage \ of \pageref{LastPage}}

	\setlength{\parindent}{0pt}

	\newcommand{\tab}[1]{\hspace{.2\textwidth}\rlap{#1}}
	
	\newcommand{\comment}[1]{}

	\newcommand{\mathcomment}[1]{\qquad\color{blue}{(#1)}}
	\newcommand{\bigsum}[2]{\sum\limits_{#1}^{#2}}

	\renewcommand{\vec}[1]{\overrightarrow{#1}}
	\newcommand{\vectorset}[1]{\\{\vec{#1_1}, \vec{#1_2}, ..., \vec{#1_k}\\}}
	\newcommand{\vectorseqtwo}[2]{#1_1\vec{#2_1} + #1_2\vec{#2_2} + ... + #1_k\vec{#2_k}}
	\newcommand\m[1]{\begin{bmatrix}#1\end{bmatrix}}
	\newcommand\real[0]{\mathbb{R}}
	\newcommand\braces[1]{\left\\{#1\right\\}}

	\newcommand{\divider}[0]{\rule{\textwidth}{0.1pt}}

\begin{document}
\onehalfspacing
\maketitle
\tableofcontents
\pagebreak
\section{2018/01/9}
\section{2018/01/11}
\section{2018/01/16}

Diagonalization

$T: R^2 \rightarrow R^2$

projection onto the line

$A = \frac{1}{2} \m{1 & -1 \\ -1 & 1}$

x + y = 0

A is diagonalizable, ie $A + P \dot D \dot P^{-1}$

where $D = \m{1 & 0 \\ 0 & 0}, P = \m{1 & 1 \\ -1 & 1}$

Let $\vec{u_1} = \m{1 \\ -1} and \vec{v_1} = \m{1 \\ 1}$

The canonical basis of $\real^2$ is $B = \braces{ \m{1 \\ 0} = \vec{i}, \m{0 \\ 1} = \vec{j} }$

Note that $B_1 = \braces{\vec{u_1}, \vec{v_1}}$ is also a basis of $\real^2$

A is the standard matrix of $T$, it is in fact the matrix of $T$ through the canonical basis of $B$

a vector $\vec{u} \in \vec{\real^2}$ has coordinates $\m21[x y]$ with respect to $B$.

The coordinates of $T(\vec{u})$ with respect to $B$ is $A \m21[x y] = A(P \m21[x_1 y_1])$ 
% TODO change to big parentheses

Let $\m21{x_1 y_1}$ be the coordinates of $\vec{u}$ with respect to $B_1$

$\vec{us} = \vec{x_i} + \vec{y_j} = x_1 \vec{u_1} + y_1 \vec{v_1} \\
\Rightarrow \m21[x y] = P \m21[x_1 y_1]$

$P^{-1}AP \m21[x_1 y_1] = D \m21[x_1 y_1] = \m21[x_1 0]$

$D$ is the matrix of the linear transformation $T$ through the basis $B_1$

\divider

\subsection{Vector Spaces}

Let $K$ be a field $\parentheses{K = \real, K = \complex}$
Let $V$ be a nonempty set
$V$ is equipped with 2 operations

Additiona - 

\begin{tabular}{@{} l l}
	Additions	& if $\vec{u} \in \vec{v}, \vec{v} \in V$, then sum $\vec{u} + \vec{v}$ is defined \\
	Scalar Multiplication & $if \vec{u} \in V, \alpha \in \real, \alpha \vec{u}$ is defined
\end{tabular}

$V$ is called a vector space (over $K$) if the following properties hold:

% TODO enumerate with A_1, A_2, A_3 ...
\begin{itemize}
	\item whenever $\vec{u}, \vec{v} \in V, \vec{u} + \vec{v} \in V$
	\item whenever $\vec{u}, \vec{v} \in V, \vec{u} + \vec{v} = \vec{v} + \vec{u}$
	\item whenever $\vec{u}, \vec{v}, \vec{w} \in V, \parentheses{\vec{u} + \vec{v}} + \vec{w} = \vec{u} + \parentheses{\vec{v} + \vec{w}}$
	\item there exists a special vector in $V$ called the zero vector, denoted by $\vec{0}$ such that whenever $\vec{u} \in V$, $\vec{u} + \vec{0} = \vec{0} + \vec{u} = \vec{u}$
	\item Given $\vec{u} \in V$, there exists $\vec{w} \in V$ such that $\vec{u} + \vec{w} = \vec{w} + \vec{u} = \vec{0}$ \\
	$\vec{w}$ is denoted by $-\vec{u}$
\end{itemize}

% TODO enumerate with S_1, S_2, ...
\begin{itemize}
	\item $\forall \alpha \in K, \forall \vec{u} \in V, \alpha vec{u} \in V$
	\item $1 \dot \vec{u} = \vec{u}, 1 \in K(K = \real), \vec{u} \in V$
	\item whenever $\alpha, \beta \in K, \vec{u} \in V, \alpha \parentheses{\beta \vec{u}} = \parentheses{\alpha \beta} \vec{u}$
	\item whenever $\alpha, \beta \in K, \vec{u} \in V, \parentheses{\alpha + \beta}\vec{u} = \alpha \vec{u} + \beta \vec{u}
	\item whenever $\alpha \in K, \vec{u}, \vec{v} \in V$, $\alpha \parentheses{\vec{u} + \vec{v}} = \alpha \vec{u} + \alpha \vec{v}$
\end{itemize}

\divider

\subsubsection{Examples}

\begin{enumerate}
	\item $V = \real^n$ is a vector space over $K = \real$
	\item let $M_{p \times q}$ be the set of all $p \times q$ matrices \\
	$M_{p \times q}$ is a vector space over $\real$
	\item Let $P$ be the set of all polynomials over $\real$ \\
	$P_1, P_2 \in P$, $\parentheses{P_1 + P_2}(x) = P_1(x) + P_2(x) \forall x \in \real$ \\
	If $\alpha \in \real \in K, \parentheses{\alpha P}(x) = alpha P(x) \forall x \in \real$
	\item Let $0$ be the function such that $0(x) = 0 \forall x$ 
\end{enumerate}

\end{document}