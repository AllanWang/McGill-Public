\documentclass[12pt]{article}
	\usepackage{hyperref, fancyhdr, setspace, enumerate, amsmath, array, keycommand, lastpage, amssymb, xcolor}
	\usepackage[margin=1 in]{geometry}
	\allowdisplaybreaks
	\hypersetup{
		%colorlinks=true, %set true if you want colored links
		linktoc=all, %set to all if you want both sections and subsections linked
		linkcolor=black, %choose some color if you want links to stand out
	}
	\author{Allan Wang} 
	\date{Last updated: \today}
	\title{MATH 223: Linear Algebra 2}
	\pagestyle{fancy}
	\lhead{MATH 223}
	\chead{\leftmark}
	\rhead{Allan Wang}
	\cfoot{Page \thepage \ of \pageref{LastPage}}

	\setlength{\parindent}{0pt}

	\newcommand{\tab}[1]{\hspace{.2\textwidth}\rlap{#1}}
	
	\newcommand{\comment}[1]{}

	\newcommand{\mathcomment}[1]{\qquad\color{blue}{(#1)}}
	\newcommand{\bigsum}[2]{\sum\limits_{#1}^{#2}}

	\newcommand{\ddef}[1]{\textcolor{blue}{#1}}

	\newkeycommand{\ccup}[sub=i=1,sup=\infty, base=A_i] {
		\bigcup_{\commandkey{sub}}^{\commandkey{sup}}\commandkey{base}
	}

	\newkeycommand{\ccap}[sub=i=1,sup=\infty, base=A_i] {
		\bigcap_{\commandkey{sub}}^{\commandkey{sup}}\commandkey{base}
	}

	\newkeycommand{\llim}[sub=n \rightarrow \infty, base=A_n] {
		\lim_{\commandkey{sub}}\commandkey{base}
	}

	\newcommand{\braces}[1]{\left\{#1\right\}}

	\newcommand{\divider}[0]{\rule{\textwidth}{0.1pt}}

\begin{document}
\onehalfspacing
\maketitle
\tableofcontents
\pagebreak
\section{2018/01/9}
\section{2018/01/11}
\section{2018/01/16}

Diagonalization

$T: R^2 \rightarrow R^2$

projection onto the line

$A = \frac{1}{2} \matrix 2 2 [1 -1 -1 1]

x + y = 0

A is diagonalizable, ie $A + P \dot D \dot P^{-1}$

where D = \matrix 1 0 0 0, P = matrix 1 1 -1 1

Let $\vec{u_1} = \m21[1 -1] and \vec{v_1} = \m21[1 1]$

The canonical basis of $\real^2$ is $B = \braces{ \m21[1 0] = \vec{i}, \m21[0 1] = \vec{j}}

Note that $B_1 = \braces{\vec{u_1}. \vec{v_1}}$ is also a basis of $\real^2$

A is the standard matrix of $T$, it is in fact the matrix of $T$ through the canonical basis of $B$

a vector $\vec{u} \in \vec{\real^2}$ has coordinates $\m21[x y]$ with respect to $B$.

The coordinates of $T(\vec{u})$ with respect to $B$ is $A \m21[x y] = A(P \m21[x_1 y_1])$ 
% TODO change to big parentheses

Let $\m21{x_1 y_1}$ be the coordinates of $\vec{u}$ with respect to $B_1$

$\vec{us} = \vec{x_i} + \vec{y_j} = x_1 \vec{u_1} + y_1 \vec{v_1} \\
\Rightarrow \m21[x y] = P \m21[x_1 y_1]$

$P^{-1}AP \m21[x_1 y_1] = D \m21[x_1 y_1] = \m21[x_1 0]$

$D$ is the matrix of the linear transformation $T$ through the basis $B_1$

\divider

\subsection{Vector Spaces}

Let $K$ be a field $\parentheses{K = \real, K = \complex}$
Let $V$ be a nonempty set
$V$ is equipped with 2 operations

Additiona - 

\begin{tabular}[@{} l l]
	Additions	& if $\vec{u} \in \vec{v}, \vec{v} \in V$, then sum $\vec{u} + \vec{v}$ is defined \\
	Scalar Multiplication & $if \vec{u} \in V, \alpha \in \real, \alpha \vec{u}$ is defined
	
\end{tabular}


\end{document}