\documentclass[12pt]{article}
	\usepackage{hyperref, fancyhdr, setspace, array, keycommand, lastpage, xcolor}
	\usepackage{enumerate, enumitem}
	\usepackage{multiaudience, environ}
	\usepackage{amsmath, amssymb, amsfonts}
	\usepackage{mathtools}
	\usepackage[margin=1 in]{geometry}
	\allowdisplaybreaks
	\hypersetup{
		linktoc=all,
		linkcolor=black,
	}
	\author{Allan Wang} 
	\date{Last updated: \today}
	\title{MATH 223: Linear Algebra 2}
	\pagestyle{fancy}
	\lhead{MATH 223}
	\chead{\leftmark}
	\rhead{Allan Wang}
	\cfoot{Page \thepage \ of \pageref{LastPage}}
	
	\SetNewAudience{compact}
	\SetNewAudience{full}
	\def\CurrentAudience{full}

	\setlength{\parindent}{0pt}

	\newcommand{\tab}[1]{\hspace{.2\textwidth}\rlap{#1}}
	
	\newcommand{\mathcomment}[0]{\quad\color{blue}} 
	
	\newcommand{\comment}[1]{}

	\renewcommand{\v}[1]{\overrightarrow{#1}}
	\newcommand{\vectorset}[1]{\{\v{#1_1}, \v{#1_2}, ..., \v{#1_k}\}}
	\newcommand{\vectorseqtwo}[2]{#1_1\v{#2_1} + #1_2\v{#2_2} + ... + #1_k\v{#2_k}}
	\newcommand\m[1]{\begin{bmatrix}#1\end{bmatrix}}
	\newcommand\mm[1]{\begin{pmatrix}#1\end{pmatrix}}

	\newcommand{\real}[0]{\mathbb{R}}
	\newcommand{\complex}[0]{\mathbb{C}}
	\renewcommand{\natural}[0]{\mathbb{N}}
	
	\newenvironment{block}[1][Label]{\underline{#1}\par}{}
	\newenvironment{proof}{\block[Proof]}{\endblock}
	\newenvironment{proposition}{\block[Proposition]}{\endblock}
	\newenvironment{lemma}{\block[Lemma]}{\endblock}
	\newenvironment{theorem}{\block[Theorem]}{\endblock}

	% \NewEnviron{examples}{
	% 	\begin{shownto}{-,compact}
	% 		\underline{Examples}
	% 		\begin{enumerate}
	% 			\BODY
	% 		\end{enumerate}
	% 	\end{shownto}
	% }
	% \newenvironment{examples}{\underline{Examples}\enumerate}{\endenumerate\divider}
	\newenvironment{examples}{\shownto{-,compact}\underline{Examples}\enumerate}{\endenumerate\divider\endshownto}

	\newenvironment{test}{\shownto{-,compact}\enumerate}{\endenumerate\endshownto}

	\newcommand{\bb}[1]{\left\{#1\right\}}
	\newcommand{\bbb}[1]{\left[#1\right]}
	\newcommand{\pp}[1]{\left(#1\right)}

	\newcommand{\divider}[0]{\textcolor{lightgray}{\rule{\textwidth}{0.1pt}}}
	
	\newcommand{\ssum}[0]{\sum_{i=1}^n}
	
	\newcommand{\sspan}[1]{\text{span}\bb{#1}}
	
	\newenvironment{eqn}{\equation\alignedat{3}}{\endalignedat\endequation}

\begin{document}

\onehalfspacing
\maketitle
\tableofcontents
\pagebreak
\section{2018/01/9}


\section{2018/01/11}

$v \subseteq \real^n$ is a subspace of $\real^n$ if \\

\begin{tabular}{@{}l l l}
	1. & $\vec{O} \in V$ & ie $V$ is non empty \\
	2. & $\vec{u} + \vec{v} \in V$ & whenever $\vec{u} \in V + \vec{\alpha} \in V$ \\
	3. & $\alpha \vec{u} \in V$ & whenever $\vec{u} \in V, \vec{\alpha} \in \real$	
\end{tabular} \\

A subspace $V$ of $\real$ has a basis

ie a family $\vectorset{u}$ of vectors in $V$ such that $\vectorset{u}$ is a spanning set of $V$

A spanning set of $V$ is a set such that every vector in $V$ is a linear combination of that set

ie whenever $\vectorseqtwo{\alpha}{u} = \vec{0}$ then $\alpha_1 = \alpha_2 = ... = \alpha_k = 0$

if $A\alpha = 0$, rank of $A$ is $k (\le n)$, where $k =$ dimension of $V$

\begin{examples}
	\item $E = \bb{\v{u} = \m{t \\ 2t + s \\ 1}, t \in \real, s \in \real} \subseteq R^3$
	* $E$ is not a subspace of $R^3$ as the 0 matrix is not included
	\item $F = \bb{\v{u} = \m{t + s \\ 2t + s' \\ 1}, t, s' \in \real} \subseteq R^3$ \\
	$\v{u} \in F \Rightarrow \m{t + s \\ 2t = s' \\ 0} = t\m{1 \\ 2 \\ 0} + s'\m{1 \\ -1 \\ 0}$ \\
	$F =$ span $\bb{\m{1 \\ 2 \\ 0}, \m{1 \\ -1 \\ 0}}$ (linearly independent)
	\item let $A = \m{1, 1 \\ 2, -1 \\ 0, 0} \rightarrow \m{1, 0 \\ 0, 1 \\ 0, 0}$ \\
	Rank$(A) = 2$: therefore $\bb{\m{1, 1 \\ 2, -1 \\ 0, 0} \m{1, 0 \\ 0, 1 \\ 0, 0}}$ is linearly independent.
\end{examples}

\section{2018/01/16}

\subsection{Diagonalization}

\(T: R^2 \rightarrow R^2\)

projection onto the line

\(A = \frac{1}{2} \m{1 & -1 \\ -1 & 1}\)

x + y = 0

A is diagonalizable, ie \(A + P \cdot D \cdot P^{-1}\)

where \(D = \m{1 & 0 \\ 0 & 0}, P = \m{1 & 1 \\ -1 & 1}\)

Let \(\v{u_1} = \m{1 \\ -1}\) and \(\v{v_1} = \m{1 \\ 1}\)

The canonical basis of \(\real^2\) is \(B = \bb{\m{1 \\ 0} = \v{i}, \m{0 \\ 1} = \v{j}}\)

Note that \(B_1 = \bb{\v{u_1}, \v{v_1}}\) is also a basis of \(\real^2\)

A is the standard matrix of \(T\), it is in fact the matrix of \(T\) through the canonical basis of \(B\)

a vector \(\v{u} \in \v{\real^2}\) has coordinates \(\m{x \\ y}\) with respect to \(B\).

The coordinates of \(T(\v{u})\) with respect to \(B\) is \(A \m{x \\ y} = A\pp{P \m{x_1 \\ y_1}}\) 

Let \(\m{x_1 \\ y_1}\) be the coordinates of \(\v{u}\) with respect to \(B_1\)

\(\v{us} = \v{x_i} + \v{y_j} = x_1 \v{u_1} + y_1 \v{v_1}\) \\
\(\Rightarrow \m{x \\ y} = P \m{x_1 \\ y_1}\)

\(P^{-1}AP \m{x_1 \\ y_1} = D \m{x_1 \\ y_1} = \m{x_1 \\ 0}\)

\(D\) is the matrix of the linear transformation \(T\) through the basis \(B_1\)

\divider

\subsection{Vector Spaces}

Let \(K\) be a field \(\pp{K = \real, K = \complex}\)
Let \(V\) be a nonempty set
\(V\) is equipped with 2 operations

\begin{tabular}{@{} l l}
	Additions	& if \(\v{u} \in \v{v}, \v{v} \in V\), then sum \(\v{u} + \v{v}\) is defined \\
	Scalar Multiplication & if \(\v{u} \in V, \alpha \in \real, \alpha \v{u}\) is defined
\end{tabular}

\(V\) is called a vector space (over \(K\)) if the following properties hold:

\begin{enumerate}[label=\(A_{\arabic*}\)]
	\item whenever \(\v{u}, \v{v} \in V, \v{u} + \v{v} \in V\)
	\item whenever \(\v{u}, \v{v} \in V, \v{u} + \v{v} = \v{v} + \v{u}\)
	\item whenever \(\v{u}, \v{v}, \v{w} \in V, \pp{\v{u} + \v{v}} + \v{w} = \v{u} + \pp{\v{v} + \v{w}}\)
	\item there exists a special vector in \(V\) called the zero vector, denoted by \(\v{0}\) such that whenever \(\v{u} \in V\), \(\v{u} + \v{0} = \v{0} + \v{u} = \v{u}\)
	\item Given \(\v{u} \in V\), there exists \(\v{w} \in V\) such that \(\v{u} + \v{w} = \v{w} + \v{u} = \v{0}\) \\
	\(\v{w}\) is denoted by \(-\v{u}\)
\end{enumerate}

\divider

\begin{enumerate}[label=\(S_{\arabic*}\)]
	\item \(\forall \alpha \in K, \forall \v{u} \in V, \alpha vec{u} \in V\)
	\item \(1 \cdot \v{u} = \v{u}, 1 \in K(K = \real), \v{u} \in V\)
	\item whenever \(\alpha, \beta \in K, \v{u} \in V, \alpha \pp{\beta \v{u}} = \pp{\alpha \beta} \v{u}\)
	\item whenever \(\alpha, \beta \in K, \v{u} \in V, \pp{\alpha + \beta}\v{u} = \alpha \v{u} + \beta \v{u}\)
	\item whenever \(\alpha \in K, \v{u}, \v{v} \in V\), \(\alpha \pp{\v{u} + \v{v}} = \alpha \v{u} + \alpha \v{v}\)s
\end{enumerate}

\divider

\begin{examples}
	\item \(V = \real^n\) is a vector space over \(K = \real\)
	\item let \(M_{p \times q}\) be the set of all \(p \times q\) matrices \\
	\(M_{p \times q}\) is a vector space over \(\real\)
	\item Let \(P\) be the set of all polynomials over \(\real\) \\
	\(P_1, P_2 \in P\), \(\pp{P_1 + P_2}(x) = P_1(x) + P_2(x) \ \forall x \in \real\) \\
	If \(\alpha \in \real \in K, \pp{\alpha P}(x) = \alpha P(x) \ \forall x \in \real\)
	\item Let \(0\) be the function such that \(0(x) = 0 \ \forall x\) 
\end{examples}

\section{2018/01/18}

\subsection{Vector Spaces}

Examples \\
Let $D$ be a subset of $\real$ ($D$ can be an interval for example) \\
Let $F(D)$ be the set of all real valued functions defined on $D$ \\
For $f, g \in F(D)$, $\alpha, \beta \in \real$, $0 : D \rightarrow \real$

\begin{itemize}
	\item $f + g : D \rightarrow \real$
	\item $(f + g)(x) = f(x) + g(x)$
	\item $(\alpha f)(x) = \alpha \cdot f(x)$
	\item $f + g = g + f$
	\item $(f + g) + h = f + (g + h)$
	\item $0(x) = 0$
	\item $f + 0 = f$
	\item $f + (-f) = 0$
	\item $1 \cdot f = f$
	\item $(\alpha + \beta) f(x) = \alpha f(x) + \beta f(x) = (\alpha f + \beta f)(x)$
\end{itemize}

Note that if we set $D = \natural$

$F(\natural) =$ set of all real-valued sequences

\subsection{Proposition}

Let $(V, +, \cdot)$ be a vector space over $K$


\begin{enumerate}
	\item The zero vector $\v{0}$ in $V$ is unique
	\item Given $\v{u} \in V$, the vector $\v{-u}$ is unique
	\item If $\alpha \v{u} = 0$ then $\alpha = 0$ or $\v{u} = \v{0}$
	\item $\v{-u} = (-1) \v{u}$
\end{enumerate}

\begin{proof}
	\begin{enumerate}
		\item Let $\v{0_1}$ and $\v{0_2}$ be two vectors such that
		
		\begin{equation}
		\begin{cases}
		\v{u} + \v{0_1} = \v{0_1} + \v{u} = \v{u} \quad \forall \v{u} \\
		\v{u} + \v{0_2} = \v{0_2} + \v{u} = \v{u} \quad \forall \v{u}
		\end{cases}
		\end{equation}
		
		It follows that \\
		$\v{0_1} = \v{0_1} + \v{0_2} = \v{0_2}$
		
		\item Let $\v{u} \in V$ and let $\v{w_1}$ and $\v{w_2}$ be two vectors such that \\
		$\v{u} + \v{w_1} = \v{0}$ \\
		$\v{u} + \v{w_2} = \v{0}$
		
		\begin{eqn}
		\v{u} + \v{w_1} &= \v{0} &\\
		\v{w_2} + (\v{u} + \v{w_1}) &= \v{w_2} + \v{0} &\\
		(\v{w_2} + \v{u}) + \v{w_1} &= \v{w_2} & \mathcomment \text{associativity}\\
		0 + \v{w_1} &= \v{w_2} &\\
		\v{w_1} &= \v{w_2} &
		\end{eqn}
		
		\item Suppose $\alpha \v{u} = 0$ 
		If $\alpha \ne 0$ \\
		
		\begin{eqn}
		\frac{1}{\alpha} \in K \quad K = \real \\
		\frac{1}{\alpha} (\alpha \v{u}) = \frac{1}{\alpha} \v{0} = \v{0} \\
		(\frac{1}{\alpha} \alpha) \v{u} = \v{0} \text{\quad ie} 1 \cdot \v{u} = \v{u} = 0
		\end{eqn}
		
		\item $-\v{u} = (-1) \v{u}$
		
		\begin{eqn}
		&& 1 + (-1) &= 0 \\
		&& (1 + (-1)) \v{u} &= 0 \v{u} = \v{u} \\
		&& 1 \v{u} + (-1) \v{u} &= \v{0} \\
		&& \v{u} + (-1) \v{u} &= \v{0} \\
		\therefore && (-1) \v{u} &= \v{-u}
		\end{eqn}
		
	\end{enumerate}
\end{proof}

\subsection{Subspaces}

Let $(V, + \cdot)$ be a vector space over $K$ \\
Let $E$ be a subset of $V (E \subseteq V)$ \\
$(E, + \cdot)$ is called a subspace of $(V, +, \cdot)$ \\
if $(E, +, \cdot)$ is a vector space over $K$.
	
\begin{proposition}
	$E$ is a subspace of $V$ if the following properties hold:
	
	\begin{enumerate}
		\item $\v{0} \in E$ 
		\item Whenever $\v{u}, \v{v} \in E \qquad \v{u} + \v{v} \in E$
		\item Whenever $\v{u} \in E, \alpha \in K \qquad \alpha \v{u} \in E$
	\end{enumerate}
\end{proposition}

Notice that $E \subseteq V$ is a subspace of $V$ iff $E$ is nonempty and $\alpha \v{u} + \beta \v{v} \in E$ whenever $\v{u}, \v{v} \in E, \alpha, \beta \in K$

\begin{examples}
	\item Let $C([0, 1])$ be the set of all continuous functions on $[0, 1]$ \\
	$C([0, 1]) \subseteq F([0, 1])  \leftarrow \text{vector space}$ \\
	The function $f: [0, 1] \rightarrow \real$ \\
	$f \in C([0, 1]) \text{(nonemptiness)}$ \\
	If $f$ and $g$ are continuous on $[0, 1]$, so if $f + g$, as well as $\alpha f \ \forall \alpha$ \\
	$C[0, 1]$ is a subspace of $F([0, 1])$
	
	\item Let $E = \bb{A \in M_{2 \times 2} \mid A = A^T}$ \\
	Note that $I_2 = \mm{1 & 0 \\ 0 & 1} \in E$ \\
	whenever $A, B \in E$, $(A + B)^T = A^T + B^T = A + B$ \\
	$A + B \in E$ \\
	Also $(\alpha A)^T = \alpha A^T = \alpha A$ \\
	ie $\alpha A \in E$ \\
	$E$ is a subspace of $M_{2 \times 2}$
\end{examples}

\section{2018/01/23}

\subsection{Subspaces}

\begin{examples}
	\item Let $E = \bb{p = P_3, \text{ such that } p(1) = 2}$ \\
	$E$ is a nonempty subset of $P_3 (p(x) = 2x \in E)$. But $E$ is neither stable under addition nor stable under scalar multiplication. \\\\
	Ex $p_1(x) = 2x \in E$, but $(4p_1)(x) = 8x \notin E$. \\
	$\therefore E$ is not a subspace 
	
	\item Let $E = \bb{p \in P_3 \mid p(0) \ge 0}$ \\
	The zero polynomial $(0) \in E$ \\
	let $p_1 \in E, p_2 \in E, (p_1 + _2)(0) = p_1(0) + p_2(0) \ge 0$ \\
	$p1 + p_2 \in E$ \\
	However, $E$ is not stable under scalar muliplication. \\
	Ex $p(x) = x + 1 \in E \leftarrow p(0) = 1 \ge 0$ \\
	if $\alpha < 0$, then $\alpha p(0) = \alpha < 0 \rightarrow \alpha p \notin E$
	
	\item If A is a $n \times m$ matrix \\
	$\text{Null}(A) = \bb{x \in \real^m \mid AX = 0}$ \\
	$\text{Null}(A)$ is a subspace of $\real^m$ \\
	\begin{proof}
		$X = 0 \in \text{Null}(A) \text{ since } A0 = 0$
		Let $X_1, X_2 \in \text{Null}(A)$ \\
		$A(X_1 + X_2) = AX_1 + AX_2 = 0 + 0 = 0$ \\
		If $X \in \text{Null}(A), \alpha \in \real$ \\
		$\alpha X \in \text{Null}(A)$ bc \\
		$A(\alpha X) = \alpha (AX) = \alpha 0 = 0$
	\end{proof}
\end{examples}

	Let $(V, +, \cdot)$ be a vector space on $\real$. \\
	Let $\v{u_1}, \v{u_2}, ..., \v{u_n}$ be $n$ vector in $V$ 
	
	\begin{proposition}
		The subset $E \subseteq V$ of all linear combinations $(lc)$ of $\v{u_1}, \v{u_2}, ..., \v{u_n}$ is a subspace of $V$, and is denoted \\
		$E = \sspan{\v{u_1}, \v{u_2}, ..., \v{u_n}}$
		
		\begin{proof}
			\begin{enumerate}
				\item $\v{0} \in E$ bc $\v{0} = 0 \v{u_1} + 0 \v{u_2} + ... + 0 \v{u_n}$
				\item $E$ is stable under addition \\
				Let $\v{v} = \sum_{i=1}^n \alpha_i \v{u_i} \in E$ \\
				$\v{w} = \sum_{i=1}^n \beta_i \v{u_1} \in E$ \\
				$\v{v} + \v{w} = \sum_{i=1}^n (\alpha_i + \beta_i) \v{u_i} \in E$
				\item $E$ is stable under scalar multiplication \\
				$\v{u} =  \ssum \alpha_i \v{u_i} \in E$ and $\beta \in \real$ \\
				$\beta \v{u} = \beta(\ssum \alpha_i \v{u_i}) = \sum_{i=1}^n (\beta \alpha_i) \v{u_i} \in E$
			\end{enumerate}
		\end{proof}
	\end{proposition}
	
	\begin{examples}
		\item Let $A$ be a $n \times m$ matrix and $C_1, C_2, ..., C_m$ are the columns of $A$ (each column $\in \real^n$). \\
		$\sspan{C_1, C_2, ..., C_m}$ is a vector subspace of $\real^n$, called the column space of $A$ and denoted $\text{Col}(A) n$. \\
		Similarly, the row space of $A$ is $\text{Row}(A) = \text{Col}(A^T)$ is a subspace of $\real^m$.
		
		\item $E = P_3$ \\
		$p \in P_3), p(x) = ax^3 + bx^2 + cx + d$ \\
		$P_3 = \sspan{x^3, x^2, x, 1}$		
		
		\item $E = \bb{p \in P_3 \mid p(2) = 0}$ is a subspace of $P_3$ \\
		If $p \in E, p(x) = 0$ \\
		ie $p(x) = (x - 2)q(x)$ where $q(x) \in P_2$ \\
		$p(x) = (x - 2)(ax^2 + bx + c) = ax^2(x - 2) + bx(x - 2) + c(x - 2) \quad a, b, c \in \real$ \\
		$E = \sspan{x^2(x - 2), x(x - 2), x - 2}$ \\
		$p \in P_3, p(x) = sum_{k=0}^3 \frac{f^{(k)}(2)}{k!} (x - 2)^k$ \\
		if $p \in E, p(2) = 0$ \\
		\begin{eqn}
			p(x) &= \sum_{k=0}^3 \frac{p^{(k)}(2)}{k!} (x - 2)^k \\
			&= \frac{p^{(1)}(2)}{1!}(x - 2) + \frac{p^{(2)}(2)}{2!}(x-2)^2 + \frac{p^{(3)}(2)}{3!}(x-2)^3
		\end{eqn}
	\end{examples}
	
	\begin{proposition}
		Let $E$ be a subspace of $V$ \\
		Let $F_1 = \bb{\v{u_1}, \v{u_2}, ..., \v{u_n}}$ be a subset of vectors in $V$ \\
		$F_2 = \bb{\v{v_1}, \v{v_2}, ..., \v{v_n}}$ be a subset of vectors in $V$ \\
		$F_1$ and $F_2$ are both spanning sets of the same subspace $E$ of $V$ iff every vector in $F_1$ is a $lc$ of vectors in $F_2$ and every vector in $F_2$ is a $lc$ of vectors in $F_1$.
	\end{proposition}

\section{2018/01/25}

Wasn't there

\section{2018/01/30}	

\subsection{Linear Independence}

\subsubsection{Properties}


% PROPERTY sets should go to n, not k
\begin{itemize}
	\item If a subset $\vectorset{u}$ of vectors in $V$ contains the zero vector $\pp{\v{u_i} = \v{0} \text{ for some } i}$, then it is linearly dependent
	
	\item If $F = \vectorset{u}$ is linearly independent, then any subset of $F$ is linearly dependent
	
	\item if $F = \vectorset{u}$ is linearly independent, and $\bb{\v{u_1}, \v{u_2}, ..., \v{u_n}, \v{u_{n+1}}}$ is linearly independent, then $\v{u_{n+1}} \in \sspan{\v{u_1}, \v{u_2}, ..., \v{u_n}}$
	
	\begin{proof}
		\begin{enumerate}
			\item Without loss of generality, $\v{u_1} = \v{0}$ \\
			Note that $2 \v{u_1} + 0 \v{u_2} + 0 \v{u_3} + ... + 0 \v{u_n} = \v{0}$ \\
			As there is a nonzero coefficient, there must be linear dependence.
			
			\item Let $F = \vectorset{u}$ be linearly independent. \\
			Let $F_1$ be a subset of $F$ containing $k$ vectors, $k \le n$ \\
%			WLOG, $F__1 = \bb{\v{u_1}, \v{u_2}, ..., \v{u_k}}$ \\
			$\sum_{i=1}^n \alpha_i \v{u_i} = \v{0} \Rightarrow \sum_{i=1}^k \alpha_i \v{u_i} + 0 \v{u_{i+1}} + 0 \v{u_{i+2}} + ... 0 \v{u_n} = \v{0}$
		\end{enumerate}
	\end{proof}

	Since $F$ is linearly independent, we must have $\alpha_1 = \alpha_2 = ... = \alpha_k = 0$

	\item Assume that $F = \vectorset{u}$ is linearly independent and $\bb{\v{u_1}, \v{u_2}, ..., \v{u_n}, \v{u_{n+1}}}$ is linearly dependent \\
	
	There exists a finite sequence $\alpha_1, \alpha_2, ..., \alpha_n, \alpha_{n+1}$, where not all values are zeroes, such that \\
	$\alpha_1 \v{u_1} + \alpha_2 \v{u_2} + ... + \alpha_n \v{u_n} + \alpha_{n+1} \v{u_{n+1}} = \v{0}(*)$ \\
	
	Claim $\alpha_{n+1} \ne 0$ \\
	Assume $\alpha_{n+1} = 0$ \\
	$\alpha_{n+1} = 0$ and $(*)$ yields \\
	$\alpha_1 \v{u_1} + \alpha_2 \v{u_2} + ... + \alpha_n \v{u_n} = \v{0}$ \\
	which implies $\alpha_1 = \alpha_2 = ... = \alpha_n = 0$ (since $F$ is linearly independent). That is a contraction, therefore $\alpha_{n+1} \ne 0$
	
	$(*)$ can be rewritten as $\alpha_{n+1} \v{u_{n+1}} = \alpha_1 \v{u_1} + \alpha_2 \v{u_2} + ... + \alpha_n \v{u_n} = \sum_{i=1}^n \alpha_i \v{u_i}$ \\
	$\v{u_{n+1}} = \sum_{i=1}^n -\pp{\frac{\alpha_i}{\alpha{n+1}}} \v{u_i}$ \qquad ie $\v{u_{n+1}} \in \sspan{\v{u_1}, \v{u_2}, ..., \v{u_n}}$
\end{itemize}

\begin{proposition}
	If $F = \vectorset{u}$ is linearly dependent, then one of the $\v{u_i}$ can be written as the linear combination of the others.
\end{proposition}

\underline{Basis} \\
Let $V$ be a vector apces and $E$ be a subspace of $V$. A basis of $E$ is a family $F = \vectorset{u}$ of vectors in $E$ such that

\begin{enumerate}
	\item $E = \sspan{\v{u_1}, \v{u_2}, ..., \v{u_n}}$
	\item $F = \vectorset{u}$ is linearly independent
\end{enumerate} 

\begin{examples}
	\item $V = \real^3$ \qquad A basis of $V$ is $\bb{\m{1 \\ 0 \\ 0}, \m{0 \\ 1 \\ 0}, \m{0 \\ 0 \\ 1}}$
	
	\item Let $V$ be any vector space \\
	$E = \bb{\v{0}}$ does not have a basis because the only spanning set if $\bb{\v{0}}$ which is linearly dependent
	
	\begin{lemma}
		Let $E$ be a subspace of $V$ \\
		Let $F_1 = \vectorset{u}$ be a spanning set of $E$ \\
		Let $F_2 = \vectorset{v}$ be a linearly independent subset of $E$ \\
		then $m \ge n$
	\end{lemma}

	\begin{proof}
		By contradiction \\
		Assume that $n > m$ \\
		$\bbb{\v{v_1}, \v{v_2}, ..., \v{v_n}} = \bbb{\v{u_1}, \v{u_2}, ..., \v{u_n}} AX$ \\
		$AX = 0$ 
		% A is m x n
		% X is n x 1
	\end{proof}

	Fn $j = 1, 2, ..., n$ \\
	$\v{v_j} = \sum_{i=1}^m a_{ij} \v{u_i}$ \qquad (because $F_1$ is a spanning set of $E$) \\
	Let  $A = \pp{a_{ij}}_{1 \le i \le m \\ 1 \le j \le n}$ \\
	% todo line break in subtext
	
	$\sum_{j=1}^m x_j \v{v_j} = \sum_{i=1}^m \pp{\sum_{j=1}^n \alpha_{ij} x_j} \v{u_i} (*)$ \\
	
	$A$ is $m \times n$ and $n > m$ \\
	Therefore, the homogeneous system $AX = 0$ has a non trivial solution. \\
	Using the components of the nontrivial solution in $(*)$, we have $\sum_{j=1}^n x_j \v{v_j} = \v{0}$, but not all $x_j$ are equal to $0$. \\
	ie  $F_2$ is linearly dependent, which is a contradiction
\end{examples}

\begin{theorem}
	Let $V$ be a vector space and $E$ be a subspace of $V$ such that $E \ne \bb{\v{0}}$ \\
	All basis of $E$ have the same number $k$ of vectors; $k$ is called the dimension of $E$ \\
	Notation \qquad $dim(E) = k$
	
	\begin{proof}
		Let $B_1 = \bb{set u1 to uk}$ and $B_2 = \bb{set v1 to vl}$ be two basis of $E$. We have to prove that $l = k$ \\
		
		\begin{equation*}
			\begin{rcases*}
				B_1 \text{ is a spanning set of } E \\
				B_2 \text{ is linearly independent in } E
			\end{rcases*} \Rightarrow k \ge l
		\end{equation*}
		
		\begin{equation*}
		\begin{rcases*}
		B_2 \text{ is a spanning set of } E \\
		B_1 \text{ is linearly independent in } E
		\end{rcases*} \Rightarrow l \ge k
		\end{equation*}
		
		$k = l$
	\end{proof}
\end{theorem}

\begin{block}[Remark]
	$E = \bb{\v{0}}$ \quad
	$dim(E) = 0$ \quad
	$dim(\real^3) = 3$
\end{block}

\begin{examples}
	\item $P_n =$ set of all polynomials of order $\le n$ \\
	We have seen that $B = \bb{1, x, x^2, ..., x^n}$ is a spanning set of $P_n$ and is also linearly independent \\
	
	$B$ is a basis of $P_n$ \\
	therefore $dim(P_n) = n + 1$
	
	\item $M_{2 \times 2} =$ set of all $2 \times 2$ matrices \\
	$E_1 = \mm{1 & 0 \\ 0 & 0}$
	$E_2 = \mm{0 & 0 \\ 1 & 0}$
	$E_1 = \mm{0 & 0 \\ 0 & 1}$
	$E_1 = \mm{0 & 1 \\ 0 & 0}$
	
	$\bb{E_1, E_2, E_3, E_4}$ is a basis of $M_{2 \times 2}$ \\
	$M = \mm{a & c \\ b & d} = aE_1 + bE_2 + dE_3 + cE_4$ \\
	$dim(M_{2 \times 2}) = 2 \times 2 = 4$
\end{examples}

\end{document}